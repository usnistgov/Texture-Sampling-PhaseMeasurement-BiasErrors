\documentclass{article}
\usepackage{amsmath}
\usepackage[numbers]{natbib}
\usepackage{geometry}
\usepackage[final]{graphicx}
\usepackage{bm}
\geometry{letterpaper,margin=1in}
\usepackage{authblk}

\bibliographystyle{alpha} % alpha for easier checking/tracking references
%\bibliographystyle{unsrt}

%%%%%%%%%%%%%%%%%%%
%%%% Comment conventions %%%%
% "[Name]" question is addressed to that person
% "Discuss" open comment about the content of the paper for us to sort out
% "Note" no action required
%%%%%%%%%%%%%%%%%%%

\begin{document}

\title{Scheme Report} % Note: placeholder
\date{\today}
\author[1, 2]{Surya Chandramouleeswaran}
\author[3]{Adam Creuziger}
\author[2]{Michael Cox}
\author[2]{Kip Findley}
\affil[1]{BASIS Chandler}
\affil[2]{Metallurgical and Materials Engineering Department, Colorado School of Mines}
\affil[3]{Materials Science and Engineering Division, Material Measurement Laboratory, NIST}


\maketitle

\section{Background}


\section{Methods}

% Software used to create schemes
% Data used for simulated phase fractions



% Considerations of the scheme
% Equations of the scheme

% Discuss: How many peaks included
% Discuss: Some of these could either be complete pole figures or partial (tilt angle limited).  Run both options?

\subsection{Schemes}
% Discuss: organize by complete/partial, or 'style'.  Currently I have them by 'style', and roughly sorted by when I think they were developed.
% Discuss: currently this is 27 different schemes. That seems like a lot...

\subsubsection{Single schemes}
% ND, TD, RD
\begin{figure}[ht]
    \centering
    \includegraphics[width=0.85\paperwidth]{NDSingle_scatter.png}
    \caption{Sample Caption}
\end{figure}

\begin{figure}[ht]
    \centering
    \includegraphics[width=0.85\paperwidth]{RDSingle_scatter.png}
    \caption{Sample Caption}
\end{figure}

\begin{figure}[ht]
    \centering
    \includegraphics[width=0.85\paperwidth]{TDSingle_scatter.png}
    \caption{Sample Caption}
\end{figure}



% Morris

\begin{figure}[ht]
    \centering
    \includegraphics[width=0.85\paperwidth]{Morris_scatter.png}
    \caption{Sample Caption}
\end{figure}


\subsubsection{Ring schemes}
% ND, TD, RD
\begin{figure}[ht]
    \centering
    \includegraphics[width=0.85\paperwidth]{RingND_scatter.png}
    \caption{Sample Caption}
\end{figure}

\begin{figure}[ht]
    \centering
    \includegraphics[width=0.85\paperwidth]{RingRD_scatter.png}
    \caption{Sample Caption}
\end{figure}

\begin{figure}[ht]
    \centering
    \includegraphics[width=0.85\paperwidth]{RingTD_scatter.png}
    \caption{Sample Caption}
\end{figure}





% rotated rings - complete and partial (2)
\begin{figure}[ht]
    \centering
    \includegraphics[width=0.85\paperwidth]{ring_scatter.png}
    \caption{Sample Caption}
\end{figure}

\begin{figure}[ht]
    \centering
    \includegraphics[width=0.85\paperwidth]{partialring_scatter.png}
    \caption{Sample Caption}
\end{figure}


% Single reflection ring (offset ring, RD/TD, 45?)
\begin{figure}[ht]
    \centering
    \includegraphics[width=0.85\paperwidth]{OffsetF_scatter.png}
    \caption{Sample Caption}
\end{figure}

\begin{figure}[ht]
    \centering
    \includegraphics[width=0.85\paperwidth]{OffsetA_scatter.png}
    \caption{Sample Caption}
\end{figure}


\subsubsection{Equal Angle}

\begin{figure}[ht]
    \centering
    \includegraphics[width=0.85\paperwidth]{EA_scatter.png}
    \caption{Sample Caption}
\end{figure}

\subsubsection{Tilt and Rotate}

% Rotation alone
\begin{figure}[ht]
    \centering
    \includegraphics[width=0.85\paperwidth]{rotation_scatter.png}
    \caption{Sample Caption}
\end{figure}


% Tilt alone
\begin{figure}[ht]
    \centering
    \includegraphics[width=0.85\paperwidth]{tilt60deg_scatter.png}
    \caption{Sample Caption}
\end{figure}


% Tilt and rotation
\begin{figure}[ht]
    \centering
    \includegraphics[width=0.85\paperwidth]{rotationtilt60deg_scatter.png}
    \caption{Sample Caption}
\end{figure}



\subsubsection{Spiral schemes}
% Rizzle (logarithmic)
\begin{figure}[ht]
    \centering
    \includegraphics[width=0.85\paperwidth]{rizzie_scatter.png}
    \caption{Sample Caption}
\end{figure}


% Klug & Alexander
\begin{figure}[ht]
    \centering
    \includegraphics[width=0.85\paperwidth]{holden_scatter.png}
    \caption{Sample Caption}
\end{figure}

% New Spiral developed
\begin{figure}[ht]
    \centering
    \includegraphics[width=0.85\paperwidth]{Spiral_scatter.png}
    \caption{Sample Caption}
\end{figure}


\subsubsection{Hexagonal schemes}
% Rizzle
\begin{figure}[ht]
    \centering
    \includegraphics[width=0.85\paperwidth]{rizziehex_scatter.png}
    \caption{Sample Caption}
\end{figure}


% Matthias (Thomas')
\begin{figure}[ht]
    \centering
    \includegraphics[width=0.85\paperwidth]{Thomas_scatter.png}
    \caption{Sample Caption}
\end{figure}


% partial hex
\begin{figure}[ht]
    \centering
    \includegraphics[width=0.85\paperwidth]{partialhex_scatter.png}
    \caption{Sample Caption}
\end{figure}


\subsubsection{CLRGrid}

\begin{figure}[ht]
    \centering
    \includegraphics[width=0.85\paperwidth]{CLR_scatter.png}
    \caption{Sample Caption}
\end{figure}


\subsubsection{Gaussian Quadrature}

\begin{figure}[ht]
    \centering
    \includegraphics[width=0.85\paperwidth]{Gauss_scatter.png}
    \caption{Sample Caption}
\end{figure}




%\subsection{Partial Coverage Schemes}


\section{Results}

% Figure - Each scheme -- can be seen above


% Figure - Oversampling contour plots -- THESE START HERE
\subsection{Contour Plot Images}

\subsubsection{Single schemes}
% ND, TD, RD
\begin{figure}[ht]
    \centering
    \includegraphics[width=0.85\paperwidth]{ND Single_contour.png}
    \caption{Sample Caption}
\end{figure}

\begin{figure}[ht]
    \centering
    \includegraphics[width=0.85\paperwidth]{RD Single_contour.png}
    \caption{Sample Caption}
\end{figure}

\begin{figure}[ht]
    \centering
    \includegraphics[width=0.85\paperwidth]{TD Single_contour.png}
    \caption{Sample Caption}
\end{figure}



% Morris

\begin{figure}[ht]
    \centering
    \includegraphics[width=0.85\paperwidth]{Morris Single_contour.png}
    \caption{Sample Caption}
\end{figure}


\subsubsection{Ring schemes}
% ND, TD, RD
\begin{figure}[ht]
    \centering
    \includegraphics[width=0.85\paperwidth]{Ring Perpendicular to ND_contour.png}
    \caption{Sample Caption}
\end{figure}

\begin{figure}[ht]
    \centering
    \includegraphics[width=0.85\paperwidth]{Ring Perpendicular to RD_contour.png}
    \caption{Sample Caption}
\end{figure}

\begin{figure}[ht]
    \centering
    \includegraphics[width=0.85\paperwidth]{Ring Perpendicular to TD_contour.png}
    \caption{Sample Caption}
\end{figure}





% rotated rings - complete and partial (2)
\begin{figure}[ht]
    \centering
    \includegraphics[width=0.85\paperwidth]{RotRing Axis-Y Res-5.0 Theta-2.84623415 OmegaMax-90_contour.png}
    \caption{Sample Caption}
\end{figure}

\begin{figure}[ht]
    \centering
    \includegraphics[width=0.85\paperwidth]{RotRing Axis-Y Res-5.0 Theta-2.84623415 OmegaMax-60_contour.png}
    \caption{Sample Caption}
\end{figure}


% Single reflection ring (offset ring, RD/TD, 45?)
\begin{figure}[ht]
    \centering
    \includegraphics[width=0.85\paperwidth]{OffsetF_contour.png}
    \caption{Sample Caption}
\end{figure}

\begin{figure}[ht]
    \centering
    \includegraphics[width=0.85\paperwidth]{OffsetA_contour.png}
    \caption{Sample Caption}
\end{figure}


\subsubsection{Equal Angle}

\begin{figure}[ht]
    \centering
    \includegraphics[width=0.85\paperwidth]{EA_contour.png}
    \caption{Sample Caption}
\end{figure}

\subsubsection{Tilt and Rotate}

% Rotation alone
\begin{figure}[ht]
    \centering
    \includegraphics[width=0.85\paperwidth]{Rotation-NoTilt_contour.png}
    \caption{Sample Caption}
\end{figure}


% Tilt alone
\begin{figure}[ht]
    \centering
    \includegraphics[width=0.85\paperwidth]{NoRotation-tilt60deg_contour.png}
    \caption{Sample Caption}
\end{figure}


% Tilt and rotation
\begin{figure}[ht]
    \centering
    \includegraphics[width=0.85\paperwidth]{Rotation-60detTilt_contour.png}
    \caption{Sample Caption}
\end{figure}



\subsubsection{Spiral schemes}
% Rizzle (logarithmic)
\begin{figure}[ht]
    \centering
    \includegraphics[width=0.85\paperwidth]{Rizzie_contour.png}
    \caption{Sample Caption}
\end{figure}


% Klug & Alexander
\begin{figure}[ht]
    \centering
    \includegraphics[width=0.85\paperwidth]{Holden_contour.png}
    \caption{Sample Caption}
\end{figure}

% New Spiral developed
\begin{figure}[ht]
    \centering
    \includegraphics[width=0.85\paperwidth]{Spiral_contour.png}
    \caption{Sample Caption}
\end{figure}


\subsubsection{Hexagonal schemes}
% Rizzle
\begin{figure}[ht]
    \centering
    \includegraphics[width=0.85\paperwidth]{Rizzie Hex_contour.png}
    \caption{Sample Caption}
\end{figure}


% Matthias (Thomas')
\begin{figure}[ht]
    \centering
    \includegraphics[width=0.85\paperwidth]{thomas_contour.png}
    \caption{Sample Caption}
\end{figure}


% partial hex
\begin{figure}[ht]
    \centering
    \includegraphics[width=0.85\paperwidth]{Partial hex_contour.png}
    \caption{Sample Caption}
\end{figure}


\subsubsection{CLRGrid}

\begin{figure}[ht]
    \centering
    \includegraphics[width=0.85\paperwidth]{CLR_contour.png}
    \caption{Sample Caption}
\end{figure}


\subsubsection{Gaussian Quadrature}

\begin{figure}[ht]
    \centering
    \includegraphics[width=0.85\paperwidth]{Gauss_contour.png}
    \caption{Sample Caption}
\end{figure}



% Discuss: how to show effect on phase fractions.  I'm starting to think showing each peak from a Cu source may be effective.  
% This would mean changing up the heat map so that X axis would be scheme, Y axis would be texture (both phases), and separate figures for each peak (14 total).
% The 'correct' value would be equal to 1, and we could use the same color map as the oversampling contour plots  

\section{Discussion}

% Comparison of each one, pros and cons.
% Table - number of points

\section{Conclusions}



\section{Explicit Citations (to be deleted soon)}
This paragraph is a placeholder to enumerate the citations that will be 
used in this report. \cite{Bruker} and \cite{Bunge} are the 2 first citations that 
come by alphabetical convention. Next come \cite{Callister}, \cite{Cox}, and \cite{Creuziger}. 
This is followed by \cite{Cullity}, \cite{Helmholtz}, \cite{Jacques}, and \cite{Matthies}. Moving 
on, we have \cite{Raabe} and \cite{Rizzie}, accompanied by 2 works by the same author: \cite{Rollett1} and \cite{Rollett2}. As 
we move towards the end of the citations, we have \cite{Schields} and \cite{Shackelford} rounding out the list of citations 
used in this report. I'm adding an extra line so that we can get the \textit{References} section on its own page.

\bibliography{biblio.bib}


\end{document}